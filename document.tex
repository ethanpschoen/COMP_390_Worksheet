\documentclass[10pt,twocolumn]{article}

% use the oxycomps style file
\usepackage{oxycomps}

% usage: \fixme[comments describing issue]{text to be fixed}
% define \fixme as not doing anything special
\newcommand{\fixme}[2][]{#2}
% overwrite it so it shows up as red
\renewcommand{\fixme}[2][]{\textcolor{red}{#2}}
% overwrite it again so related text shows as footnotes
%\renewcommand{\fixme}[2][]{\textcolor{red}{#2\footnote{#1}}}

% read references.bib for the bibtex data
\bibliography{references}

% include metadata in the generated pdf file
\pdfinfo{
    /Title (Git and LaTeX Worksheet)
    /Author (Justin Li)
}

% set the title and author information
\title{Git and \LaTeX Worksheet}
\author{Justin Li}
\affiliation{Occidental College}
\email{justinnhli@oxy.edu}

\begin{document}

\maketitle

\section{Instructions}

This worksheet is due March 1, 2025 at midnight, to be submitted as a GitHub repository URL to Canvas. The repository should contain all files requires to compile this worksheet with your answers. You should only change this \texttt{document.tex} file and the  \texttt{references.bib} file; do not change any other file in this starting repository. You should not use any additional packages, and are not allowed to use the \texttt{{\textbackslash}usepackage\{\}} command. Additionally, the output should be formatted correctly: your answers should be appropriately nested under the questions, command-line commands should be in monospace, and images should be positioned appropriately.

\section{Git Questions}

\subsection{General questions}

\begin{enumerate}
    \item What is a version control system? Why are they useful?
    \item What is the difference between git and GitHub?
    \item What is a repository?
    \item What is a commit?
    \item What is the commit graph?
    \item What is your preferred local git client (eg., command line, GitHub Desktop, GitKraken, etc.)?
\end{enumerate}

\subsection{Local Usage}

\begin{enumerate}
\item What is the difference between adding a file to the staging area and committing a file?
\item What is a commit message, and why is it important for them to be meaningful?
\item Starting with an empty repository, what sequence of commands/actions would result in the following commit graph? You may give a sequence of \texttt{git} commands, or describe (with screenshots) how you would do this in your preferred graphical git interface.
\begin{verbatim}
A---B---C---D
\end{verbatim}
\item If you are currently at commit D above, how would you recover code from commit B? What sequence of commands/actions would let you do so? You may give a sequence of \texttt{git} command-line commands, or describe (with screenshots) how you would do this in your preferred graphical git interface. Assume the commit hashes are AAAAAA..., BBBBBB..., etc.
\item Imagine you created a git repository for your project, but only commit your changes once a week on Sundays. You got your code working on Tuesday, but then broke your code on Friday. What can you do to get the working version of your code back?
\end{enumerate}

\subsection{Branching and Merging}

\begin{enumerate}
\item What is a branch? Why are they useful?
\item Starting with an empty repository, what sequence of commands/actions would result in the following commit graph? You may give a sequence of \texttt{git} command-line commands, or describe (with screenshots) how you would do this in your preferred graphical git interface.
\begin{verbatim}
A---B---C---D
     \
      E---F
\end{verbatim}
\item Why is a merge? Why are they useful?
\item Imagine you are currently at commit D above. What sequence of commands/actions would result in the following commit graph? You may give a sequence of \texttt{git} commands, or describe (with screenshots) how you would do this in your preferred graphical git interface.
\begin{verbatim}
A---B---C---D---G
     \         /
      E---F---/
\end{verbatim}
\item What is a merge conflict? When do they occur?
\item Starting with an empty repository, despite a sequence of commands/actions that would result in a merge conflict. Include the exact edits and \texttt{git} commands or screenshots of the graphical git interface. Include the output or screenshot that shows the resulting merge conflict.
\end{enumerate}

\subsection{Remotes}

\begin{enumerate}
\item What is a remote?
\item What does pushing and pulling do?
\item Imagine you created a git repository for your project on your laptop and commit regularly, but only push your code to GitHub once a week on Sundays. Your laptop caught on fire on Friday. What can you do to get your code back?
\end{enumerate}

\section{\LaTeX}

Find a source of each of the following types and add it to \texttt{references.bib}, with the appropriate data. Your sources do not have to relate to your project. Looking at \textcite{OverleafBibliographyManagement} and \textcite{WikipediaBibtex} may be helpful,

\begin{itemize}
\item a journal article
\item a conference article
\item a PhD or Master's thesis
\item an article in an edited popular media venue (newspaper, magazine, etc.)
\item a book
\item a chapter of a book
\item a YouTube video
\item a piece of technical documentation (e.g., a programming language reference, and API documentation, etc.)
\end{itemize}

Additionally, in you own words, explain the difference between \texttt{{\textbackslash}cite\{\}} and \texttt{{\textbackslash}textcite\{\}}. When should they each be used? Demonstrate your answers by using one of them with each of your references from above.

\printbibliography

\end{document}
